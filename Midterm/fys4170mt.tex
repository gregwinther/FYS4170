\documentclass[11pt, a4paper]{amsart}

\usepackage[utf8]{inputenc}
\usepackage[]{hyperref}
\usepackage[]{amsmath}
\DeclareMathOperator{\Tr}{Tr}
\usepackage[]{bbm}
\usepackage[]{physics}
\usepackage[]{graphicx}

\title[FYS4170: Take-home Exam]{Take-home Exam \\
   \hrulefill \small{ FYS4170: Quantum Field Theory }\hrulefill}
   
 \author[15020]{Candidate number: 15020}
 \date{\today}
 
 \begin{document}
 
\maketitle

\section{Trace of Dirac Matrix Products}

The Dirac gamma matrices are in the chiral representation given in $2\times 2$ block form as,

\begin{equation}
\gamma^0 = \begin{pmatrix}
0 & 1 \\ 1 & 0
\end{pmatrix}, \quad
\gamma^i = \begin{pmatrix}
0 & \sigma^i \\ \sigma & 0
\end{pmatrix},
\end{equation}

where $\sigma^i$ are the Pauli sigma matrices,

\begin{equation}
\sigma^1 = \begin{pmatrix}
0 &  1 \\ 1 & 0
\end{pmatrix},
\quad
\sigma^2 = \begin{pmatrix}
0 & -i \\ i & 0
\end{pmatrix},
\quad
\sigma^3 = \begin{pmatrix}
1 & 0 \\ 0 & -1
\end{pmatrix}.
\end{equation}

The gamma matrices satisfy the anti-commutation relations

\begin{equation}
\label{eq:gamma_anti_comm}
\{\gamma^\mu, \gamma^\nu\} \equiv \gamma^\mu \gamma^\nu = 2g^{\mu\nu} \times \mathbbm{1},
\end{equation}

The fifth gamma matrix is defined by
\begin{equation}
\gamma^5 \equiv i \gamma^0 \gamma^1 \gamma^2 \gamma^3 
= -\frac{i}{4!}\epsilon^{\mu\nu\rho\sigma}\gamma_\mu\gamma_\nu\gamma_\rho\gamma_\sigma
\end{equation}

and has the following properties

\begin{align}
(\gamma^5)^\dagger &= \gamma^5, \label{eq:hermitian_gamma}\\
(\gamma^5)^2 &= 1, \label{eq:gamma5sq} \\
\{\gamma^5, \gamma^\mu \} &= 0 \label{eg:gamma5anti_comm}.
\end{align}

\subsection*{a}

The trace of the product of two gamma matrices can be evaluated using the anti-commutator relation and the cyclic property of the trace of a matrix product,

\begin{align}
\Tr[\gamma^\mu \gamma^\nu] &= \Tr[2g^{\mu\nu} \times \mathbbm{1} - \gamma^\nu\gamma^\mu] \\
   &= 2g^{\mu\nu} \Tr \mathbbm{1} - \Tr[\gamma^\nu\gamma^\mu] \label{eq:g_is_elem} \\
   &= 8g^{\mu\nu} - \Tr[\gamma^\mu\gamma^\nu] \label{eq:trace_cyclic1}.
\end{align}
In this particular case $g^{\mu\nu}$ is a matrix element, not the metric, and can therefore be moved outside the trace operator as in \ref{eq:g_is_elem}. In \ref{eq:trace_cyclic1} the cyclic property of the trace is employed. This yields

\begin{equation}
\Tr[\gamma^\mu\gamma^\nu] = 4g^{\mu\nu}.
\end{equation}

\subsection*{b}

The trace of the product of two gamma matrices with the fifth gamma matrix is,
\begin{align}
\Tr[\gamma^\mu\gamma^\nu\gamma^5] 
  &= \ \ \ \Tr[\gamma^0\gamma^0\gamma^\mu\gamma^\nu\gamma^5] \\
  &= -\Tr[ \gamma^0\gamma^5\gamma^0 \gamma^\mu \gamma^\nu]  \\
  &= -\Tr[\gamma^\mu\gamma^\nu\gamma^0\gamma^5\gamma^0]  \\
  &= -\Tr[\gamma^\mu\gamma^\nu(\gamma^5)^\dagger] \\
  &= -\Tr[\gamma^\mu\gamma^\nu\gamma^5] .
\end{align}
Here we have first inserted $\mathbbm{1} = \gamma^0\gamma^0$, then the anticommutator relation from \autoref{eg:gamma5anti_comm} is employed three times. The cyclic property of the trace is employed and finally $\gamma^0\gamma^\mu\gamma^0 = \gamma^\dagger$ and the fact that $\gamma^5$ is Hermitian (\autoref{eq:hermitian_gamma}). Adding $\Tr[\gamma^\mu\gamma^\nu\gamma^5]$ to both sides gives $2\Tr[\gamma^\mu\gamma^\nu\gamma^5] = 0$ and we have that

\begin{equation}
\Tr|\gamma^\mu\gamma^\nu\gamma^5] = 0.
\end{equation}

\subsection*{c}

The trace of an odd number of gamma matrices is always zeros. Here is a proof for three gamma matrices

\begin{align}
\Tr[\gamma^\mu\gamma^\nu\gamma^\rho] 
  &=\ \ \ \Tr[\gamma^5\gamma^5\gamma^\mu\gamma^\nu\gamma^\rho] \\
  &= -\Tr[\gamma^5\gamma^\mu\gamma^\nu\gamma^\rho\gamma^5] \\
  &= -\Tr[\gamma^5\gamma^5\gamma^\mu\gamma^\nu\gamma^\rho] \\
  &= -\Tr[\gamma^\mu\gamma^\nu\gamma^\rho],
\end{align}
where first the property in \autoref{eq:gamma5sq} is employed, then three anticommutations from \autoref{eg:gamma5anti_comm} and the cyclic property of the trace. Adding $\Tr[\gamma^\mu\gamma^\nu\gamma^\rho]$ similar to before gives $2\Tr[\gamma^\mu\gamma^\nu\gamma^\rho] = 0$ and we end up with,

\begin{equation}
\Tr[\gamma^\mu\gamma^\nu\gamma^\rho] = 0.
\end{equation}

This proof holds for any odd number of gamma matrices because the number of commutation relation ``switches'' needed will then also be odd, yielding the desired minus sign.

\subsection*{d}

The trace of the product of four gamma matrices and the special $\gamma^5$ is zero in \emph{nearly} all cases. In fact, this is the first non-vanishing trace involving $\gamma^5$. Let us first try the same kind of trick as before,

\begin{align}
\Tr[\gamma^\mu\gamma^\nu\gamma^\rho\gamma^\sigma\gamma^5] 
   &= \ \ \ \Tr[\gamma^0\gamma^0\gamma^\mu\gamma^\nu\gamma^\rho\gamma^\sigma\gamma^5] \\
   &= -\Tr[\gamma^0\gamma^5\gamma^0\gamma^\mu\gamma^\nu\gamma^\rho\gamma^\sigma] \\
   &= -\Tr[\gamma^\mu\gamma^\nu\gamma^\rho\gamma^\sigma\gamma^0\gamma^5\gamma^0] \\
   &= -\Tr[\gamma^\mu\gamma^\nu\gamma^\rho\gamma^\sigma\gamma^5],
\end{align}

which in the same way as in the previous cases yields

\begin{equation}
\Tr[\gamma^\mu\gamma^\nu\gamma^\rho\gamma^\sigma\gamma^5] = 0.
\end{equation}

However, the result is something else if all of Dirac's gamma matrices are represented in the trace. Take for instance $(\mu\nu\rho\sigma) = (0123)$,

\begin{align}
\Tr[\gamma^0\gamma^1\gamma^2\gamma^3\gamma^5]
   &= \ \ i\Tr[\gamma^0\gamma^1\gamma^2\gamma^3\gamma^0\gamma^1\gamma^2\gamma^3] \\
   &= - i\Tr[\gamma^0\gamma^0\gamma^1\gamma^2\gamma^3\gamma^1\gamma^2\gamma^3] \\
   &= - i\Tr[\gamma^1\gamma^1\gamma^2\gamma^3\gamma^2\gamma^3] \\
   &= \ \ i\Tr[\gamma^2\gamma^3\gamma^2\gamma^3] \\
   &= - i\Tr[\gamma^2\gamma^2\gamma^3\gamma^3] \\
   &= - i\Tr[\mathbbm{1}] = -i4.
\end{align}

A computation with the indices of two adjacent gamma matrices intechanged.

\begin{align}
\Tr[\gamma^0\gamma^1\gamma^3\gamma^2\gamma^5]
   &= \ \ i\Tr[\gamma^0\gamma^1\gamma^3\gamma^2\gamma^0\gamma^1\gamma^2\gamma^3] \\
   &= - i\Tr[\gamma^0\gamma^0\gamma^1\gamma^3\gamma^2\gamma^1\gamma^2\gamma^3] \\
   &= - i\Tr[\gamma^1\gamma^1\gamma^3\gamma^2\gamma^2\gamma^3] \\
   &= \ \ i\Tr[\gamma^3\gamma^2\gamma^2\gamma^3] \\
   &= - i\Tr[\gamma^3\gamma^3] \\
   &= \ \  i\Tr[\mathbbm{1}] = i4.
\end{align}

The initial ordering of the for gamma matrices will change the sign compared to the initial case, because the number of commutation relation switches needed to complete the computation will change as well. This means that if two adjacent indices are interchanged, the sign will change. If two indices with another index between them are changed the sign stays the same - equivalent to two adjacent index exchanges. In other words, an even number of permutations will leave the sign unchanged, while an odd number of permutations will not. For all other cases, where two or more of the indices are equal the answer is zero. The result must therefore be proportional to the four-dimensional Levi-Civita symbol, as well as $-i4$ from the trial computation. In conclusion,

\begin{equation}
\Tr[\gamma^\mu\gamma^\nu\gamma^\rho\gamma^\sigma\gamma^5] = -i4\epsilon^{\mu\nu\rho\sigma}.
\end{equation}

\section{Scattering Cross Section}

A central result in quantum field theory is

\begin{equation}
\label{eq:infinitesimal_cross_section}
d\sigma = \frac{1}{4vE_\mathcal{A}E_\mathcal{B}} \abs{\mathcal{M}}^2 d\Pi_n,
\end{equation}

\subsection*{a}
In \autoref{eq:infinitesimal_cross_section}, $d\sigma$ is the infinitesimal cross section, whereas the cross section is defined by 
\begin{equation}
\sigma \equiv \frac{\text{Number of scattering events}}{\rho_\mathcal{A}\ell_\mathcal{A}\rho_\mathcal{B}\ell_\mathcal{B} A},
\end{equation}
for two colliding bunches of particles $A$  and $B$. $\rho_\mathcal{A}$ and $\rho_\mathcal{B}$ are the densities of the particle bunches, $\ell_\mathcal{A}$ and $\ell_\mathcal{B}$ are the lengths of the bunches of particles, and $A$ is the cross-sectional area common to the two bunches. The cross-section can be thought of as the probability of scattering events. It has area as unit, as it gives a measure of an area that two particles need to be within to interact. When colliding two beams of particles with well-defined momenta, one can express the likelihood of any particular final state in terms of the cross section. The relative velocity of the two particle bunches is given by $v = \abs{v_\mathcal{A} - v_\mathcal{B}}$ in the lab frame, while $E_\mathcal{A}$ and $E_\mathcal{B}$ are the energies of the particle beams.

$\mathcal{M}$ is the matrix element of the interesting part of the $S$(cattering) matrix ($S = \mathbbm{1} + i T$) after the four-momentum conservation term is factored out,

\begin{equation}
\bra{\{\vb{p}_f\}}iT\ket{\{\vb{k}_i\}} 
\equiv {}_{\text{out}}\braket{\{\vb{p}_f\}}{\{\vb{k}_i\}}_\text{in}
= i\mathcal{M}(2\pi)^4 \delta^{(4)}(\Sigma k_i - \Sigma p_f),
\end{equation}

where $\{\vb{p}_i$ and $\{\vb{p}_f\}$ are the sets of \emph{in} and \emph{out} states, respectively. $\mathcal{M}$ is interesting because it describes the actual scattering. If the particles in question does not interact at all, $S$ is simply the identity operator - the final state will be equal to the initial state.

Lastly, the infinitesimal relativistic invariant $n$-body final-state momenta is (a tongue twister) given by

\begin{equation}
d\Pi_n = \left(\frac{d^3p_f}{(2\pi)^3} \frac{1}{2E_f}\right) (2\pi)^4 \delta^(4).
\end{equation}

It contains both a normalising factor (????) and the four-momentum conservation factor.

\subsection*{b}

\section{Calculation $\mathcal{M}$, Feynman Rules}

\subsection*{a}

\subsection*{b}

\section{Axial Vector Current Conservation}



\end{document}