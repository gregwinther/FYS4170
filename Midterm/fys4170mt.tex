\documentclass[11pt, a4paper]{amsart}

\usepackage[utf8]{inputenc}
\usepackage[]{hyperref}
\usepackage[]{amsmath}
\DeclareMathOperator{\Tr}{Tr}
\usepackage[]{bbm}
\usepackage[]{physics}
\usepackage[]{graphicx}

\title[FYS4170: Take-home Exam]{Take-home Exam \\
   \hrulefill \small{ FYS4170: Quantum Field Theory } \hrulefill}
   
 \author[Candidate numer]{Candidate number: [?????]}
 \date{\today}
 
 \begin{document}
 
\maketitle

\section{Trace of Dirac Matrix Products}

The Dirac gamma matrices are in the chiral representation given in $2\times 2$ block form as,

\begin{equation}
\gamma^0 = \begin{pmatrix}
0 & 1 \\ 1 & 0
\end{pmatrix}; \quad
\gamma^i = \begin{pmatrix}
0 & \sigma^i \\ \sigma & 0
\end{pmatrix},
\end{equation}

where $\sigma^i$ are the Pauli sigma matrices,

\begin{equation}
\sigma^1 = \begin{pmatrix}
0 &  1 \\ 1 & 0
\end{pmatrix},
\quad
\sigma^2 = \begin{pmatrix}
0 & -i \\ i & 0
\end{pmatrix},
\quad
\sigma^3 = \begin{pmatrix}
1 & 0 \\ 0 & -1
\end{pmatrix}.
\end{equation}

The gamma matrices satisfy the anti-commutation relations

\begin{equation}
\label{eq:gamma_anti_comm}
\{\gamma^\mu, \gamma^\nu\} \equiv \gamma^\mu \gamma^\nu = 2g^{\mu\nu} \times \mathbbm{1}.
\end{equation}

\subsection*{a}

The trace of the product of two gamma matrices can be evaluated using the anti-commutator relation and the cyclic property of the trace of a matrix product,

\begin{align}
\Tr[\gamma^\mu \gamma^\nu] &= \Tr[2g^{\mu\nu} \times \mathbbm{1} - \gamma^\nu\gamma^\mu] \\
   &= 2g^{\mu\nu} \Tr \mathbbm{1} - \Tr[\gamma^\nu\gamma^\mu] \label{eq:g_is_elem} \\
   &= 8g^{\mu\nu} - \Tr[\gamma^\mu\gamma^\nu] \label{eq:trace_cyclic1}.
\end{align}
In this particular case $g^{\mu\nu}$ is a matrix element, not the metric, and can therefore be moved outside the trace operator as in \ref{eq:g_is_elem}. In \ref{eq:trace_cyclic1} the cyclic property of the trace is employed. This yields

\begin{equation}
\Tr[\gamma^\mu\gamma^\nu] = 4g^{\mu\nu}.
\end{equation}

 \end{document}