\documentclass[11pt, a4paper]{amsart}

\usepackage[utf8]{inputenc}
\usepackage[]{hyperref}
\usepackage[]{amsmath}
\DeclareMathOperator{\Tr}{Tr}
\usepackage[]{slashed}
\usepackage[]{bbm}
\usepackage[]{physics}
\usepackage[]{graphicx}
\usepackage[]{cancel}
\usepackage[]{tikz}
\usepackage[]{tikz-feynman}
\usepackage[offset=1.2em]{simpler-wick}
\usepackage{bm}

\title[FYS4170: Take-home Exam]{Take-home Exam \\
   \hrulefill \small{ FYS4170: Quantum Field Theory }\hrulefill}
   
 \author[15020]{Candidate number: 15020}
 \date{\today}
 
 \begin{document}
 
\maketitle

\section{Trace of Dirac Matrix Products}

The Dirac gamma matrices are in the chiral representation given in $2\times 2$ block form as,

\begin{equation}
\gamma^0 = \begin{pmatrix}
0 & 1 \\ 1 & 0
\end{pmatrix}, \quad
\gamma^i = \begin{pmatrix}
0 & \sigma^i \\ -\sigma^i & 0
\end{pmatrix},
\end{equation}

where $\sigma^i$ are the Pauli sigma matrices,

\begin{equation}
\sigma^1 = \begin{pmatrix}
0 &  1 \\ 1 & 0
\end{pmatrix},
\quad
\sigma^2 = \begin{pmatrix}
0 & -i \\ i & 0
\end{pmatrix},
\quad
\sigma^3 = \begin{pmatrix}
1 & 0 \\ 0 & -1
\end{pmatrix}.
\end{equation}

The gamma matrices satisfy the anti-commutation relations

\begin{equation}
\label{eq:gamma_anti_comm}
\{\gamma^\mu, \gamma^\nu\} \equiv \gamma^\mu \gamma^\nu = 2g^{\mu\nu} \times \mathbbm{1},
\end{equation}

The fifth gamma matrix is defined by
\begin{equation}
\gamma^5 \equiv i \gamma^0 \gamma^1 \gamma^2 \gamma^3 
= -\frac{i}{4!}\epsilon^{\mu\nu\rho\sigma}\gamma_\mu\gamma_\nu\gamma_\rho\gamma_\sigma
\end{equation}

and has the following properties

\begin{align}
(\gamma^5)^\dagger &= \gamma^5, \label{eq:hermitian_gamma}\\
(\gamma^5)^2 &= 1, \label{eq:gamma5sq} \\
\{\gamma^5, \gamma^\mu \} &= 0 \label{eg:gamma5anti_comm}.
\end{align}

\subsection*{a}

The trace of the product of two gamma matrices can be evaluated using the anti-commutator relation and the cyclic property of the trace of a matrix product,

\begin{align}
\Tr[\gamma^\mu \gamma^\nu] &= \Tr[2g^{\mu\nu} \times \mathbbm{1} - \gamma^\nu\gamma^\mu] \\
   &= 2g^{\mu\nu} \Tr \mathbbm{1} - \Tr[\gamma^\nu\gamma^\mu] \label{eq:g_is_elem} \\
   &= 8g^{\mu\nu} - \Tr[\gamma^\mu\gamma^\nu] \label{eq:trace_cyclic1}.
\end{align}
In this particular case $g^{\mu\nu}$ is a matrix element, not the metric, and can therefore be moved outside the trace operator as in \ref{eq:g_is_elem}. In \ref{eq:trace_cyclic1} the cyclic property of the trace is employed. This yields

\begin{equation}
\Tr[\gamma^\mu\gamma^\nu] = 4g^{\mu\nu}.
\end{equation}

\subsection*{b}

The trace of the product of two gamma matrices with the fifth gamma matrix is,
\begin{align}
\Tr[\gamma^\mu\gamma^\nu\gamma^5] 
  &= \ \ \ \Tr[\gamma^0\gamma^0\gamma^\mu\gamma^\nu\gamma^5] \\
  &= -\Tr[ \gamma^0\gamma^5\gamma^0 \gamma^\mu \gamma^\nu]  \\
  &= -\Tr[\gamma^\mu\gamma^\nu\gamma^0\gamma^5\gamma^0]  \\
  &= -\Tr[\gamma^\mu\gamma^\nu(\gamma^5)^\dagger] \\
  &= -\Tr[\gamma^\mu\gamma^\nu\gamma^5] .
\end{align}
Here we have first inserted $\mathbbm{1} = \gamma^0\gamma^0$, then the anticommutator relation from \autoref{eg:gamma5anti_comm} is employed three times. The cyclic property of the trace is employed and finally $\gamma^0\gamma^\mu\gamma^0 = \gamma^\dagger$ and the fact that $\gamma^5$ is Hermitian (\autoref{eq:hermitian_gamma}). Adding $\Tr[\gamma^\mu\gamma^\nu\gamma^5]$ to both sides gives $2\Tr[\gamma^\mu\gamma^\nu\gamma^5] = 0$ and we have that

\begin{equation}
\Tr|\gamma^\mu\gamma^\nu\gamma^5] = 0.
\end{equation}

\subsection*{c}

The trace of an odd number of gamma matrices is always zeros. Here is a proof for three gamma matrices

\begin{align}
\Tr[\gamma^\mu\gamma^\nu\gamma^\rho] 
  &=\ \ \ \Tr[\gamma^5\gamma^5\gamma^\mu\gamma^\nu\gamma^\rho] \\
  &= -\Tr[\gamma^5\gamma^\mu\gamma^\nu\gamma^\rho\gamma^5] \\
  &= -\Tr[\gamma^5\gamma^5\gamma^\mu\gamma^\nu\gamma^\rho] \\
  &= -\Tr[\gamma^\mu\gamma^\nu\gamma^\rho],
\end{align}
where first the property in \autoref{eq:gamma5sq} is employed, then three anticommutations from \autoref{eg:gamma5anti_comm} and the cyclic property of the trace. Adding $\Tr[\gamma^\mu\gamma^\nu\gamma^\rho]$ similar to before gives $2\Tr[\gamma^\mu\gamma^\nu\gamma^\rho] = 0$ and we end up with,

\begin{equation}
\Tr[\gamma^\mu\gamma^\nu\gamma^\rho] = 0.
\end{equation}

This proof holds for any odd number of gamma matrices because the number of commutation relation ``switches'' needed will then also be odd, yielding the desired minus sign.

\subsection*{d}

The trace of the product of four gamma matrices and the special $\gamma^5$ is zero in \emph{nearly} all cases. In fact, this is the first non-vanishing trace involving $\gamma^5$. Let us first try the same kind of trick as before,

\begin{align}
\Tr[\gamma^\mu\gamma^\nu\gamma^\rho\gamma^\sigma\gamma^5] 
   &= \ \ \ \Tr[\gamma^0\gamma^0\gamma^\mu\gamma^\nu\gamma^\rho\gamma^\sigma\gamma^5] \\
   &= -\Tr[\gamma^0\gamma^5\gamma^0\gamma^\mu\gamma^\nu\gamma^\rho\gamma^\sigma] \\
   &= -\Tr[\gamma^\mu\gamma^\nu\gamma^\rho\gamma^\sigma\gamma^0\gamma^5\gamma^0] \\
   &= -\Tr[\gamma^\mu\gamma^\nu\gamma^\rho\gamma^\sigma\gamma^5],
\end{align}

which in the same way as in the previous cases yields

\begin{equation}
\Tr[\gamma^\mu\gamma^\nu\gamma^\rho\gamma^\sigma\gamma^5] = 0.
\end{equation}

However, the result is something else if all of Dirac's gamma matrices are represented in the trace. Take for instance $(\mu\nu\rho\sigma) = (0123)$,

\begin{align}
\Tr[\gamma^0\gamma^1\gamma^2\gamma^3\gamma^5]
   &= \ \ i\Tr[\gamma^0\gamma^1\gamma^2\gamma^3\gamma^0\gamma^1\gamma^2\gamma^3] \\
   &= - i\Tr[\gamma^0\gamma^0\gamma^1\gamma^2\gamma^3\gamma^1\gamma^2\gamma^3] \\
   &= - i\Tr[\gamma^1\gamma^1\gamma^2\gamma^3\gamma^2\gamma^3] \\
   &= \ \ i\Tr[\gamma^2\gamma^3\gamma^2\gamma^3] \\
   &= - i\Tr[\gamma^2\gamma^2\gamma^3\gamma^3] \\
   &= - i\Tr[\mathbbm{1}] = -i4.
\end{align}

A computation with the indices of two adjacent gamma matrices intechanged.

\begin{align}
\Tr[\gamma^0\gamma^1\gamma^3\gamma^2\gamma^5]
   &= \ \ i\Tr[\gamma^0\gamma^1\gamma^3\gamma^2\gamma^0\gamma^1\gamma^2\gamma^3] \\
   &= - i\Tr[\gamma^0\gamma^0\gamma^1\gamma^3\gamma^2\gamma^1\gamma^2\gamma^3] \\
   &= - i\Tr[\gamma^1\gamma^1\gamma^3\gamma^2\gamma^2\gamma^3] \\
   &= \ \ i\Tr[\gamma^3\gamma^2\gamma^2\gamma^3] \\
   &= - i\Tr[\gamma^3\gamma^3] \\
   &= \ \  i\Tr[\mathbbm{1}] = i4.
\end{align}

The initial ordering of the for gamma matrices will change the sign compared to the initial case, because the number of commutation relation switches needed to complete the computation will change as well. This means that if two adjacent indices are interchanged, the sign will change. If two indices with another index between them are changed the sign stays the same - equivalent to two adjacent index exchanges. In other words, an even number of permutations will leave the sign unchanged, while an odd number of permutations will not. For all other cases, where two or more of the indices are equal the answer is zero. The result must therefore be proportional to the four-dimensional Levi-Civita symbol, as well as $-i4$ from the trial computation. In conclusion,

\begin{equation}
\Tr[\gamma^\mu\gamma^\nu\gamma^\rho\gamma^\sigma\gamma^5] = -i4\epsilon^{\mu\nu\rho\sigma}.
\end{equation}

\section{Scattering Cross Section}

A central result in quantum field theory is

\begin{equation}
\label{eq:infinitesimal_cross_section}
d\sigma = \frac{1}{4vE_\mathcal{A}E_\mathcal{B}} \abs{\mathcal{M}}^2 d\Pi_n,
\end{equation}

\subsection*{a}
In \autoref{eq:infinitesimal_cross_section}, $d\sigma$ is the infinitesimal cross section, whereas the cross section is defined by 
\begin{equation}
\sigma \equiv \frac{\text{Number of scattering events}}{\rho_\mathcal{A}\ell_\mathcal{A}\rho_\mathcal{B}\ell_\mathcal{B} A},
\end{equation}
for two colliding bunches of particles $A$  and $B$. $\rho_\mathcal{A}$ and $\rho_\mathcal{B}$ are the densities of the particle bunches, $\ell_\mathcal{A}$ and $\ell_\mathcal{B}$ are the lengths of the bunches of particles, and $A$ is the cross-sectional area common to the two bunches. The cross-section can be thought of as the probability of scattering events. It has area as unit, as it gives a measure of an area that two particles need to be within to interact. When colliding two beams of particles with well-defined momenta, one can express the likelihood of any particular final state in terms of the cross section. The relative velocity of the two particle bunches is given by $v = \abs{v_\mathcal{A} - v_\mathcal{B}}$ in the lab frame, while $E_\mathcal{A}$ and $E_\mathcal{B}$ are the energies of the particle beams.

$\mathcal{M}$ is the matrix element of the interesting part of the $S$(cattering) matrix ($S = \mathbbm{1} + i T$) after the four-momentum conservation term is factored out,

\begin{equation}
\bra{\{\vb{p}_f\}}iT\ket{\{\vb{k}_i\}} 
\equiv {}_{\text{out}}\braket{\{\vb{p}_f\}}{\{\vb{k}_i\}}_\text{in}
= i\mathcal{M}(2\pi)^4 \delta^{(4)}(\Sigma k_i - \Sigma p_f),
\end{equation}

where $\{\vb{p}_i$ and $\{\vb{p}_f\}$ are the sets of \emph{in} and \emph{out} states, respectively. $\mathcal{M}$ is interesting because it describes the actual scattering. If the particles in question does not interact at all, $S$ is simply the identity operator - the final state will be equal to the initial state.

Lastly, the infinitesimal relativistic invariant $n$-body final-state momenta phase space is (a tongue twister) given by

\begin{equation}
\label{eq:phasespace}
d\Pi_n = \left(\frac{d^3p_f}{(2\pi)^3} \frac{1}{2E_f}\right) (2\pi)^4 \delta^{(4)}(\Sigma p_i - \Sigma p_f).
\end{equation}

It contains both the four-momentum conservation factor and a product of infinitesimal 3-momenta of the outgoing states.

\subsection*{b}

We want to relate the scattering probability to the transition amplitudes between some set of ``in-states'' and ``out-states''. We denote $N$ as the number of scattering events, $n_\mathcal{B}$ is the number density of the wavepacket of $\mathcal{B}$-type particles traveling towards $\mathbb{A}$-type particles at rest in the lab frame. We look at a single target particle $N_\mathcal{A}$, while there may be several incoming particles $N_\mathcal{A}$. The cross section can be written as an integral over all impact parameters $b$ and the probability $\mathcal{P}$ of a scattering event
\begin{equation}
\sigma = \int d^2b \mathcal{P}(\mathcal{A}, \mathcal{B} \to 1, 2, \dots) = \int d^2b \abs{\braket{\phi_1\phi_2\dots}{\phi_\mathcal{A}\phi_\mathcal{B}}}^2,
\end{equation}

where the probability is the interesting part. Moreover,  the initial state can be written as
\begin{equation}
\ket{\phi_A\phi_B}_{\text{in}} = \int \frac{d^3 k_A}{(2\pi)^3} \int\frac{d^3k_B}{(2\pi)^3} \frac{\phi_A(\vb{k}_A)\phi_B(\vb{k}_B)e^{-i\vb{b}\vb{k}_B}}{\sqrt{4E_AE_B}} \ket{\vb{k}_A\vb{k}_B}.
\end{equation}

We express the number of scattering events $N$ as an integral over transition probabilities,
\begin{equation}
N = \frac{N_B}{A_eff} \int d^2b \prod_f \left( \int \frac{d3^p_f}{(2\pi)^3} \frac{1}{2E_f}\right) \abs{\braket{\{\vb{p}_f\}}{\phi_A\phi_B}}^2.
\end{equation}

Finally we can construct the infinitesimal scattering amplitude by combining all equations mentioned in this section.

\begin{align}
d\sigma =& \prod_f \left( \frac{d^3p_f}{(2\pi)^3} \frac{1}{2E_f} \right) \int d^2b \left(\prod_{i = A,B}\int \frac{d^3k_i}{(2\pi)^3} \frac{\phi_i(\vb{k}_i)}{\sqrt{2E_i}} \int \frac{d^3\bar{k}_i}{(2\pi)^3} \frac{\phi_i^*(\bar{\vb{k}}_i)}{\sqrt{2\bar{E}_i}} \right) \\
&\times e^{i\vb{b}(\bar{\vb{k}_B - \vb{k}_B)}}(_{\text{out}}\braket{\{\vb{p}_f\}}{\{\vb{k}_i\}}_{\text{in}})(_{\text{out}}\braket{\{\vb{p}_f\}}{\{\bar{\vb{k}}_i\}}_{\text{in}})^*
\end{align}

\section{Calculating $\mathcal{M}$, Feynman Rules}

\subsection*{a}

The invariant amplitude $\mathcal{M}$ is given by the formula
\begin{equation}
\bra{\vb{p}_1 \vb{p}_2 \dots} iT \ket{\vb{k}_{\mathcal{B}}\vb{k}_{\mathcal{A}}} = i \mathcal{M} (2\pi)^4 \delta^{(4)}(\Sigma k_i -\Sigma  k_f)
= \bra{f}T\left\{\text{exp}{-i\int dt H_I} \right\} \ket{i},
\end{equation}
where $T$ is the time-ordering operator and $H_I$ is the interaction Hamiltonian.

The general algorithm to calculate the quantity $\mathcal{M}$ from a given interaction Hamiltonian in momentum space is as follows. First, one expands the exponential term between the bra and ket in a series of the interaction factor in the interaction Hamiltonian. The expressions of higher orders will represent interactions that are of lower probability, and we usually disregard anything above order two\footnote{At least in this course.} The couple of next step is to apply Wick's theorem, transferring the time ordering into a normal orderes sum of all possible contractions.

Second, one contracts external states with external operators. These contractions represents external legs in a Feynman diagram, namely incoming or outgoing particles. 

Third, contract remaining internal operators with each other, if possible.  These contractions correspond to propagators, or internal lines in a Feynman diagram.

Fourth, draw Feynman diagrams. The two preceding steps is all that is necessary to do so. For each vertex, internal line (propegator) and external line, there are associated Feynman rules, or factors, which have been calculated during the development of quantum field theory over the past 60 years. The rules vary for what kind of interaction and what kind of particles we are modeling (photons, fermions, QED, $\phi^4$ etc). We now also disregard any unconnected diagrams, as these are (in our simple processes) representations of interactions with identical initial and final states i.e. does not contribute to $T$ matrix. Moreover, we \emph{amputate} the connected diagrams, for example to remove parts of a diagram with propagators connected to the same leg in a ``loop''. These parts have nothing to do with the actual interaction. This essentially enforces momentum conservation and removes the factor $(2\pi)^4 \delta^{(4)}(\Sigma k_i -\Sigma  k_f)$.

Fifth, integrate over undetermined momentum (if in momentum space) and divide by symmetry factor. The symmetry factor arises from the fact that there may be several ways to draw Feynman diagrams, analogous to several ways to do the same contractions.

\subsection*{b}

We are given the following interaction Lagrangian

\begin{equation}
\mathcal{L} \supset \mathcal{L}_{\text{Int}} = - \kappa \phi_a^2\phi_b^4.
\end{equation}

The interaction Hamiltonian is given as $H_I = -\int d^3x \mathcal{L}_i$ and we therefore have our scattering amplitude

\begin{equation}
i\mathcal{M} \hat{=} \bra{f}T \left\{\text{ext} \left(- i\int d^4x \kappa \phi_a^2\phi_b^4\right) \right\} \ket{i}.
\end{equation}

The first order term of the series expansion of the exponential in $\kappa$ including contractions is,

\begin{equation}
\wick{\langle \c1{\vb{p}_1} \c2{\vb{p}_2} \c3{\vb{p}_3}  \c4{\vb{p}_4} | \c5{\phi_a} \c6{\phi_a} \c1 \phi_b \c2 \phi_b \c3 \phi_b \c4 \phi_b | \c5{\vb{k}_1} \c6{\vb{k}_2} \rangle} 
= 
\feynmandiagram[baseline=(a.base), horizontal=a to o] {
                            a -- [fermion] o -- [fermion] b,
                            c -- [fermion] o -- [fermion] d,
                            e -- [charged scalar] o -- [charged scalar] f
};
\end{equation}

Above is just one of several possible Feynman diagrams describing the interactions. A vertex with six external legs as this can be drawn in $2!4!$ different ways. There are $4!$ possible ways of pairing $\phi_b$ with $\vb{p}_f$ and $2!$ possible ways of pairing $\phi_a$ with $\vb{k}_i$. This gives rise to the Feynman rule,

\begin{equation}
-i48\kappa.
\end{equation}


\section{Current Conservation and Charge of a Dirac field}
The axial vector current is defined by

\begin{equation}
j_A^\mu = \bar{\psi} \gamma^\mu \gamma^5 \psi.
\end{equation}

The current is conserved if the divergence is zero.

\begin{align}
\partial_\mu j_A^\mu 
&= (\partial_\mu \bar{\psi} ) \gamma^\mu \gamma^5 \psi + \bar{\psi} \gamma^\mu \gamma^5 (\partial_\mu \psi) \\
&= (\partial_\mu \bar{\psi} ) \gamma^\mu \gamma^5 \psi  - \bar{\psi} \gamma^5 \gamma^\mu (\partial_\mu \psi) \label{eq:current1}.
\end{align}

It is not immediately clear when this expression is zero. A step on the way is to state the Lagrangian and find the Euler-Lagrange equations. The Lagrangian for the Lagrangian is

\begin{equation}
\mathcal{L} = \bar{\psi}(i \slashed{\partial} - m) \psi =  \bar{\psi}i\gamma^\mu \partial_\mu \psi - \bar{\psi}m\psi.
\end{equation}

The Euler-Lagrange equations for $\bar{\psi}$ is

\begin{align}
\partial_\mu \left(\frac{\mathcal{L}}{\partial (\partial_\mu \bar{\psi})} \right) - \frac{\partial \mathcal{L}}{\partial \bar{\psi}} &= 0 \\
0 - (i\gamma^\mu\partial_\mu\psi - m \psi) &= 0	\\
\gamma^\mu\partial_\mu \psi = -im\psi& \label{eq:psibarEQM},
\end{align}

and for $\psi$,

\begin{align}
\partial_\mu \left(\frac{\mathcal{L}}{\partial (\partial_\mu \bar{\psi})} \right) - \frac{\partial \mathcal{L}}{\partial \bar{\psi}} &= 0 \\
\partial_\mu i\bar{\psi} \gamma^\mu - (-m\bar{\psi}) = 0 \\
\partial_\mu \bar{\psi} \gamma^\mu = im\bar{\psi}& \label{eq:psiEQM}.
\end{align}

Inserting \autoref{eq:psibarEQM} and \autoref{eq:psiEQM} into \autoref{eq:current1} yields

\begin{equation}
\partial_u j^\mu_A = im\bar{\psi}\gamma^5\psi + \bar{\psi}\gamma^5 im \psi = 2im\bar{\psi}\gamma^5 \psi,
\end{equation}

and we see that if $m=0$ that the axial vector current  must be conserved. In order to calculate the charge, we split the axial vector current up into a left-handed and right-handed part by a linear combination.

\begin{equation}
\label{eq:rightandleftcurrent}
j_A^\mu = j_R^\mu - j_L^\mu = \bar{\psi} \gamma^\mu\left(\frac{1 + \gamma^5}{2} \right)\psi - \bar{\psi} \gamma^\mu \left( \frac{1 - \gamma^5}{2}\right) \psi = \bar{\psi}\gamma^\mu\gamma^5\psi.
\end{equation}

These left- and right-handed current are individually  conserved. The Dirac equation in terms of left-handed and right-handed fields ($\psi_L$ and $\psi_R$) is

\begin{equation}
(i\gamma^\mu \partial_\mu - m ) \psi = \begin{pmatrix}
-m & i(\partial_0 + \vb{\sigma} \cdot \nabla) \\
i(\partial_0 - \vb{\sigma} \cdot \nabla) & -m
\end{pmatrix}
\begin{pmatrix}
\psi_L \\ \psi_R
\end{pmatrix}
= 0 
\end{equation}

We see that when $m=0$, we get a set of decoupled equations
\begin{align}
i(\partial_0 - \vb{\sigma} \cdot \nabla) \psi_L &= 0 \\
i(\partial_0 + \vb{\sigma} \cdot \nabla) \psi_R &= 0.
\end{align}

It will therefore be alright to compute the charge from the right- and left-handed current from \autoref{eq:rightandleftcurrent} separetely.  To be precise, one can look at one component of the Weyl spinor representation of $\psi$ at a time and employ the following eigenvalue relation

\begin{equation}
\gamma^5 \psi_{L/R} = \mp \psi_{L/R}.
\end{equation}

We get
\begin{align}
j_L^\mu &= \begin{pmatrix}
\bar{\psi}_L & \bar{\psi_R}
\end{pmatrix}\gamma^\mu \left(\frac{1 - \gamma^5}{2}\right) 
\begin{pmatrix}
\psi_L, \\
\psi_R
\end{pmatrix} = \bar{\psi}_L \gamma^\mu \psi_L \\
j_R^\mu &= \begin{pmatrix}
\bar{\psi}_L & \bar{\psi_R}
\end{pmatrix}\gamma^\mu \left(\frac{1 + \gamma^5}{2}\right) 
\begin{pmatrix}
\psi_L \\
\psi_R
\end{pmatrix} = \bar{\psi}_R \gamma^\mu \psi_R.
\end{align}

Computing the charge in parts, the left- and right-handed fields separately, gives the same result as computing the conjoined axial charge
\begin{align*}
Q_A &= Q_R - Q_L = \int d^3x j_R^0 - int d^3x j_L^0 = \int d^3x \bar{\psi}_R\gamma^0\psi_R - \int d^3x \bar{\psi}_L\gamma^0\psi_L \\
	&= \int d^3x \psi^\dagger_R\gamma^0\psi_R - \int d^3x \psi^\dagger_L\gamma^0\psi_L 
	  = \int d^3x \left[\psi^\dagger\left(\frac{1 + \gamma^5}{2} \right) \psi - \psi^\dagger\left(\frac{1 - \gamma^5}{2} \right) \psi  \right] \\
	&= \frac{1}{2} \int d^3x \left[\psi^\dagger 1 \psi - \psi^\dagger 1 \psi + 2\psi^\dagger \gamma^5 \psi \right] 
	   = \int d^3x \psi^\dagger \gamma^5 \psi.
\end{align*}

The axial charge becomes
\begin{align}
Q_A 	&= \int d^3x \psi^\dagger \gamma^5 \psi \\
 		&= \int d^3 \int \frac{d^3p}{(2\pi)^3} \int \frac{d^3q}{(2\pi)^2} \frac{1}{2\sqrt{E_{\vb{p}}} E_{\vb{q}}} \\
 		 &\times \sum_s \left(a_{\vb{p}}^su^s(p)e^{-ip\cdot x} + b_{\vb{p}}^{s\dagger}v^s(p) e^{ip\cdot x}\right)^\dagger
 		 \gamma^5
 		 			 \sum_r\left(a_{\vb{q}}^r u^r(q) e^{-ip\cdot x} + b_{\vb{q}}^{r\dagger}v^r(q)e^{iq\cdot x} \right) \nonumber \\
 		&= \int d^3x \int \frac{d^3p}{(2\pi)^3} \int \frac{d^3q}{(2\pi)^2} \frac{1}{2\sqrt{E_{\vb{p}}} E_{\vb{q}}} 
 		\sum_{rs} \Big( a_{\vb{p}}^{s\dagger}a_{q}^{s\dagger} u^{s\dagger}(p) \gamma^5 u^r(q) e^{ix\cdot (p-q)}\\
 		&\ + b_{\vb{p}}^s b_{\vb{q}}^{r\dagger}v^{s\dagger}(p) \gamma^5v^r(q)e^{ix\cdot (q-p)}
 		     + a_{\vb{p}}^{s\dagger}b_{\vb{q}}^{r\dagger} u^{s\dagger}(p) \gamma^5 v^r(q) e^{ix\cdot(p+q)}
 		     + b_{\vb{p}}^s a_{\vb{q}}^r v^{s\dagger}(p) \gamma^5 u^r(q) e^{ix\cdot (p+q)}\Big) \nonumber  \\
 		&= \int \frac{d^3p}{(2\pi)^3} \int d^3q \frac{1}{2\sqrt{E_{\vb{p}}} E_{\vb{q}}} 
 		\sum_{rs} \Big( a_{\vb{p}}^{s\dagger}a_q^{s\dagger} u^{s\dagger}(p) \gamma^5 u^r(q) e^{it(E_{\vb{p}} - E_{\vb{q}})} \delta^3(\vb{p} - \vb{q})\\
 		&\ + b_{\vb{p}}^s b_{\vb{q}}^{r\dagger}v^{s\dagger}(p) \gamma^5v^r(q)e^{it(E_{\vb{p}} - E_{\vb{q}})} \delta^3(\vb{p} - \vb{q})
 		     + a_{\vb{p}}^{s\dagger}b_{\vb{q}}^{r\dagger} u^{s\dagger}(p) \gamma^5 v^r(q) e^{it (E_{\vb{p}} + E_{\vb{q}})} \delta^3(\vb{p} + \vb{q}) \nonumber \\
 		 &\  + b_{\vb{p}}^s a_{\vb{q}}^r v^{s\dagger}(p) \gamma^5 u^r(q) e^{-it (E_{\vb{p}} + E_{\vb{q}})} \delta^3(\vb{p} + \vb{q})\Big) \nonumber \\
 		 &= \int \frac{d^3p}{(2\pi)^3 2E_{\vb{p}}} \sum_{rs} \Big( a_{\vb{p}}^{s\dagger}a_p^{s\dagger} u^{s\dagger}(p) \gamma^5 u^r(p) 
 		    + b_{\vb{p}}^s b_{\vb{p}}^{r\dagger}v^{s\dagger}(p) \gamma^5v^r(p))
 		    + a_{\vb{p}}^{s\dagger}b_{\vb{-p}}^{r\dagger} u^{s\dagger}(\vb{p}) \gamma^5 v^r(-\vb{p}) \nonumber \\
 	 &\    + b_{\vb{p}}^s a_{-\vb{p}}^r v^{s\dagger}(\vb{p}) \gamma^5 u^r(-\vb{p})  \Big).
\end{align}

Now for some intermittent computations,

\begin{align*}
u^{s\dagger} \gamma^5 u^r &= \begin{pmatrix}
\xi^{s\dagger} \sqrt{p\cdot \sigma} & \xi^{s\dagger} \sqrt{p\cdot \bar{\sigma}} 
\end{pmatrix}
\begin{pmatrix}
-1 & 0 \\
0 & 1
\end{pmatrix}
\begin{pmatrix}
\xi^{s\dagger} \sqrt{p\cdot \sigma} \\ \xi^{s\dagger} \sqrt{p\cdot \bar{\sigma}}
\end{pmatrix} \\
&= (-p\cdot\sigma + p\cdot \bar{\sigma}) \delta^{rs} = (-E_{\vb{p}} + \vb{p} \cdot \bm{\sigma} + E_{\vb{p}} -
+ \vb{p} \cdot \bm{\sigma}) \delta^{rs} \\
&= 2 \vb{p} \cdot \bm{\sigma}  \delta^{rs} \\
v^{s\dagger} \gamma^5 v^r & = 2 \vb{p} \cdot \bm{\sigma}  \delta^{rs} \\
u^{s\dagger}(\vb{p}) \gamma^5 v^r(-\vb{p}) &= \begin{pmatrix}
\xi^{s\dagger} \sqrt{p\cdot \sigma} & \xi^{s\dagger} \sqrt{p\cdot \bar{\sigma}} 
\end{pmatrix}
\begin{pmatrix}
-1 & 0 \\
0 & 1
\end{pmatrix}
\begin{pmatrix}
\sqrt{\tilde{p}\cdot\sigma} \eta^r \\
-\sqrt{\tilde{p}\cdot\sigma} \eta^r
\end{pmatrix} \\
&=(-p\cdot\sigma + p\cdot\bar{\sigma}) \xi^{s\dagger} \eta^r
= (-E_{\vb{p}} - \vb{p} \cdot \bm{\sigma} + E_{\vb{p} + \vb{p} \cdot} \bm{\sigma}) \xi^{s\dagger} \eta^r \\
&= 0 \\
v^{s\dagger}(\vb{p}) \gamma^5 u^r(-\vb{p}) &= 0,
\end{align*}

which leaves us with

\begin{equation}
Q_A = \int \frac{d^3p}{(2\pi)^3} \frac{\vb{p}\cdot \bm{\sigma}}{E_{\vb{p}}} \sum_s \left(a_{\vb{p}}^{s\dagger}a_{\vb{p}}^s + b_{\vb{p}}^s b_{\vb{p}}^{s\dagger} \right) = \int \frac{d^3p}{(2\pi)^3} \frac{\vb{p}\cdot \bm{\sigma}}{E_{\vb{p}}} \sum_s \left(a_{\vb{p}}^{s\dagger}a_{\vb{p}}^s - b_{\vb{p}}^{s\dagger} b_{\vb{p}}^{s} \right).
\end{equation}

The dot product of the three-momentum and a vector of the Pauli matrices, as seen in this expression, appears some other places, like in the Pauli-Schödinger Hamiltonian. A Hamiltonian has energy eigenvalues, which allows one to interpret this dot product as the energy. The two terms should cancel, leaving us with the ``normal'' charge,

\begin{equation}
Q_A = Q = \int \frac{d^3p}{(2\pi)^3}  \sum_s \left(a_{\vb{p}}^{s\dagger}a_{\vb{p}}^s - b_{\vb{p}}^{s\dagger} b_{\vb{p}}^{s} \right).
\end{equation}

\section{Complex Scalar Fields and ``Photons''}

The interaction of a complex scalar field $\phi$ is interacting with massless bosons (``photons'') $B^\gamma$. The interaction is described by the following interaction Lagrangian,

\begin{equation}
\mathcal{L}_{\text{Int}} = g^2 B_\mu B^\mu \abs{\phi}^2 + ig B_\mu (\phi\partial^\mu\phi^* - \phi^* \partial^\mu \phi).
\end{equation}

\subsection*{a}

We wish to compute the amplitude of $\phi\phi^* \rightarrow \gamma\gamma$, two scalar particles annihilating into two photons, to lowest order in $g$.

\begin{equation}
i\mathcal{M} \hat{=} \bra{\gamma\gamma}T\left\{\text{exp}\left(-i\int_{-T}^T dt H_I \right) \right\} \ket{\phi\phi^*}_{\substack{\text{connected,}\\ \text{amputated}}} 
\end{equation}

where $H_I = -\int d^x \mathcal{L}_{\text{Int}}$. So,

\begin{align*}
i \mathcal{M} = &\bra{\gamma\gamma} T \left\{\text{exp} \left( i \int d^4x g^2 B_\mu B^\mu \phi^* \phi + ig B_\mu (\phi \partial^\mu \phi^* - \phi^*\partial^\mu \phi) \right) \right\} \ket{\phi\phi^*} \\
	= &\bra{\gamma\gamma}T \left\{\text{ext}(i\int d^4xg^2B_\mu B^\mu \phi\phi^*)\ \text{ext} (ig \int d^4xB_\mu \phi \partial^\mu \phi^*)\ \text{ext} (-ig \int d^4xB_\mu \phi^* \partial^\mu \phi) \right\}\ket{\phi\phi^*}.
\end{align*}

Performing an expansion to lowest order of the exponential factors in $g$ yields,

\begin{align*}
i\mathcal{M} &= \bra{\gamma\gamma} T \big\{
		\left(\mathbbm{1} + ig^2 \int d^4x B_\mu B^\mu\phi\phi^* + \dots \right)
		\left(\mathbbm{1} + ig  \int d^4x B_\mu \phi \partial^\mu \phi^* + \dots \right) \\
		& \quad \times		
		\left(\mathbbm{1} - ig   \int d^4x B_\mu \phi \partial^\mu \phi^* + \dots  \right)		
		\big\}
		\ket{\phi\phi^*} \\
		&= \bra{\gamma\gamma} \mathbbm{1} \ket{\psi\psi^*} 
		+ \wick{\langle \c1 \gamma \c2 \gamma | ig^2 \int d^4x \c1 B_\mu \c2 B^\mu \c3 \phi^* \c4 \phi | \c4 \psi \c3 \psi^* \rangle } \\
		&\ \ + \wick{\langle\c1\gamma\c2\gamma|g^2 \int d^4x \c2 B_\mu \phi\partial^\mu \c3 \phi^* \c1 B_\nu \c4 \phi^* \c5 \partial^\nu \c3 \phi |\c5 \psi \c4 \psi^*\rangle} \\
		&\ \ + \bra{\gamma\gamma}T\big\{ ig \int d^4x B_\mu \phi \partial \phi^* \big\} \ket{\psi\psi^*} \\
		&\ \ + \bra{\gamma\gamma}T\big\{ -ig \int d^4x B_\mu \phi^* \partial \phi^ \big\} \ket{\psi\psi^*}  + \dots.
\end{align*} 

The two last terms will not be fully contracted and is therefore ignored, terms that are in higher of second order in $g$ are also ignored. The two contracted terms leads to the following three Feynman diagrams.

\begin{align*}
i\mathcal{M} =&
\feynmandiagram [vertical=f1 to f2, large, baseline=(current bounding box.center)] {
	i2 [particle=$\phi$]-- [charged scalar, edge label=$p_1$] o -- [charged scalar, edge label=$p_2$] i1 [particle=$\phi^*$],
	o -- [photon, momentum=$k_1$] f1 [particle=$B_\mu$], 
	o -- [photon, momentum=$k_2$] f2 [particle=$B_\nu$],
};
+ 
\feynmandiagram [vertical=o to p, baseline=(current bounding box.center)] { 
 	i1 [particle=$\phi$] -- [charged scalar, edge label=$p_1$] o -- [photon, momentum=$k_1$] f1 [particle=$B_\mu$],
 	p -- [charged scalar, edge label=$p_2$] i2 [particle=$\phi^*$],
 	o -- [charged scalar, edge label=$p_1-k_1$] p,
 	p -- [photon, momentum=$k_2$] f2 [particle=$B_\nu$],	
 	f1 -- [draw = none] f2,
 	i1 -- [draw = none] i2,
}; \\
&+ \qquad
\feynmandiagram [vertical=o to p, baseline=(current bounding box.center)] { 
 	i1 [particle=$\phi$] -- [charged scalar, edge label=$p_1$] o -- [photon, momentum=$k_2$] f1 [particle=$B_\nu$],
 	p -- [charged scalar, edge label=$p_2$] i2 [particle=$\phi^*$],
 	o -- [charged scalar, edge label=$p_1-k_2$] p,
 	p -- [photon, momentum=$k_1$] f2 [particle=$B_\mu$],	
 	f1 -- [draw = none] f2,
 	i1 -- [draw = none] i2,
};
\end{align*}

Evaluating the Feynman diagrams with the provided rules yields

\begin{align}
i\mathcal{M} =& i\mathcal{M}_\text{I} + i\mathcal{M}_\text{II} + i\mathcal{M}_\text{III} \label{eq:mikkemus}\\
	=&  2ig^2 g^{\mu\nu} \epsilon_\mu^*\epsilon_\nu^* \nonumber \\
	 &+ (-ig (p_1 + p_1 - k_1))^\mu \frac{i}{(p_1 + k_1)^2 + m^2} (-ig(p_1 - k_1 - p_2)^\nu \epsilon_\mu^*(k_1)\epsilon_\nu^*(k_2) \nonumber\\
	 &+ (-ig (p_1 + p_1 - k_2))^\mu \frac{i}{(p_1 + k_2)^2 + m^2} (-ig(p_1 - k_2 - p_2)^\nu \epsilon_\mu^*(k_2)\epsilon_\nu^*(k_1). \nonumber
\end{align}

These equations can be simplified by employing conservation of momentum,

\begin{align}
i\mathcal{M}_{\text{I}} \ \ &= 2ig^2 g^{\mu\nu} \epsilon_\mu^*(k_1)\epsilon_\nu^* \\
i\mathcal{M}_{\text{II}} \  &= i(-ig)^2 \frac{(2p_1 - k_1)^\mu (k_2 - 2p_2)^\nu}{(p_1 - k_1)^2 - m^2} \epsilon_\mu^*(k_1)\epsilon_\nu^*(k_2) \\
i\mathcal{M}_{\text{III}}   &= i(-ig)^2 \frac{(2p_1 - k_2)^\mu (k_1 - 2p_2)^\nu}{(p_1 - k_2)^2 - m^2} \epsilon_\mu^*(k_2)\epsilon_\nu^*(k_1).
\end{align}

By definition, the product of the photon polarization vector with its related photon momentum is zero, i.e. $\epsilon_\mu^*(k_2)k_2^\mu = 0$. This simplifies two of the above expressions

\begin{align}
i\mathcal{M}_{\text{II}} \ &= 4ig^2 \frac{p_1^\mu p_2^\nu}{(p_1 - k_1)^2 - m^2}\epsilon_\mu^*(k_1) \epsilon_\nu^*(k_2) \\
i\mathcal{M}_{\text{III}}  &= 4ig^2 \frac{p_1^\mu p_2^\nu}{(p_1 - k_2)^2 - m^2}\epsilon_\mu^*(k_2) \epsilon_\nu^*(k_1),
\end{align}

in turn concatenating \autoref{eq:mikkemus} to 
\begin{equation}
i\mathcal{M} = 4ig^2 \left(\frac{1}{2}g^{\mu\nu} + \frac{p_1^\mu p_2^\nu}{(p_1 - k1)^2 - m^2} + \frac{p_1^\nu p_2^\mu}{(p_1- k_2)^2 - m^2} \right) \epsilon_\mu^*(k_1) \nu^*(k_2). 
\end{equation}

Notice the index change necessary to make this work.

The remainding goal of this venture is to use Ward's identity to show that we have found the correct expression for $i\mathcal{M}$. Ward's identity can be stated as 
\begin{equation}
p_\mu \mathcal{M}^\mu = 0,
\end{equation}
where $p_\mu$ is the momentum of an external photon in the interaction process, in a simple form stated as $\mathcal{M} = \epsilon_\mu \mathcal{M}^\mu$. This identity is essentially a statement of current conservation, which is a consequence of the gauge symmetry of QED\footnote{$\psi(x) \to e^{i\alpha(x)}\psi(x)$, $A_\mu \to A_\mu - \frac{1}{e}\partial_\mu \alpha(x)$.}. The amplitude vanishes when the polarization is replaced by the momentum.

Replacing $\epsilon_\mu^*(k_1)$ with $k_1$, one ``diagram'' at a time gives
\begin{align}
k_{1\mu}\mathcal{M}_{\text{I}}^\mu \ \ &= 2g^2k_1^\nu\epsilon_\nu^*(k_2) \\
k_{1\mu}\mathcal{M}_{\text{II}}^\mu \  &= (ig)^2\frac{(2p_1^\mu k_{1\mu} - k_1^2)(k_2^\nu\epsilon_\nu^*(k_2) - 2p_2^\nu\epsilon_\nu^*(k_2))}{k_1 - 2p_1k_1} \label{eq:didstuff1} \\
 &= g^2(k_2\epsilon^*(k_2) - 2p_2\epsilon^*(k_2)) \nonumber,
\end{align}

wherein \autoref{eq:didstuff1} we done the following replacement, $(p_1 - k_1)^2 - m^2 = k_1^2 - 2k_1p_1 + p_1 - m^2 = k_1^2 - 2k_1p_1$. Employing a similar trick like this and making use of conservation of momentum to change variables, gives us an alternative expression for the last term,

\begin{equation}
i \mathcal{M}_{\text{III}} = i(ig)^2 \frac{(2p_2 - k_1)^\mu(k_2 - 2p_1)}{p_3^2 - 2p_3p_2} \epsilon^*_\mu(k_1) \epsilon^*_\nu(k_2).
\end{equation}

Exchanging the polarization vector with the momentum vector yields,
\begin{align}
k_{1\mu}\mathcal{M}_{\text{III}} &= (ig)^2 \frac{(2p_2^\mu k_{1\mu} - k_1^2)(k_2^\nu\epsilon^*_\nu(k_2) - 2p_1^\nu\epsilon^*_\nu(k_2))}{k_1^2 - 2k_1p_2} \\
		&= g^2(k_2\epsilon^*(k_2) - 2p_1 \epsilon^*(k_2)). \nonumber
\end{align}

Finally, adding up all the terms,
\begin{align}
k_{1\mu}\mathcal{M}^\mu &= 2g^2k_1\epsilon^*(k_2) + g^2(k_2\epsilon^*(k_2) - 2p_2\epsilon^*(k_2)) + g^2(k_2\epsilon^*(k_2) - 2p_1\epsilon^*(k_2)) \\
		&= 2g^2\epsilon^*(k_2) ( k_1 + k_2 - p_1 - p_2) = 0,
\end{align}

we see that we indeed have Ward's identity!

\subsection*{b}

From the problem sets in this course we have the differential cross section

\begin{equation}
\frac{d\sigma}{d\Omega} = \frac{1}{64\pi^2} \frac{\abs{p_f}}{p_i} \abs{\mathcal{M}}^2.
\end{equation}

The absolute square scattering amplitude is

\begin{equation}
\abs{\mathcal{M}}^2 = \sum_i=1^2 \sum_j=1^2 \abs{\epsilon_i^{*\mu}\epsilon_j^{*\nu}\mathcal{M}_\mu\nu}.
\end{equation}

To be able to solve this I guess one would need to use the Ward identity to simplify, or rather apply the Feynman problem solving algorithm;
\begin{enumerate}
\item Write down problem,
\item Think very hard,
\item Write solution.
\end{enumerate}
I was, regrettably, unsuccessful.

\section{Compton Scattering}
The expression for the spin-averaged squared matrix element can be simplified to
\begin{equation}
\abs{\bar{\mathcal{M}}}^2 = 2e^4 \left[\frac{p\cdot k'}{p\cdot k} + \frac{p \cdot k}{p \cdot k'} + \left(1 + \frac{m_e^2}{p\cdot k} - \frac{m_e^2}{p\cdot k'} \right) - 1 \right]
\end{equation}

\subsection*{a}

In the lab frame, where the electron is initially at rest, we have the following momenta:
\begin{align}
k  &= (\omega, 0, 0, \omega) \\
p  &= (m, 0, 0, 0) \\
k' &= (\omega', \omega' \sin \theta, 0, \omega' \cos \theta) \\
p' &= (E', \vb{p}')
\end{align}

where $\omega$ is the photon energy and $\theta$ is the scattering angle of the photon. In $\abs{\mathcal{M}}^2$, one can contract the four-momenta to
\begin{align}
p \cdot k \   &= m \omega \label{eq:dot1}\\
p \cdot k' &= m \omega' \label{eq:dot2}
\end{align}

\subsection*{b}
We now want to express $\omega'$ in terms of $\omega$ and $\theta$. Remembering the Lorentz invariance of the square of the four momentum ($p^2 = p'^2 = m^2$) and employing conservation of momentum we get
\begin{equation}
m^2 = (p')^2 = (p + k - k')^2 = p^2 + 2p(k-k') + (k -k')^2 = p^2 + 2p(k-k') + \cancel{k^2} - 2k\cdot k' + \cancel{k'^2}
\end{equation}
employing \autoref{eq:dot1} and \autoref{eq:dot2}, computing $k\cdot k'$ and that $p^2 = m^2$ gives
\begin{align}
  m^2 &= m^2 + 2m(\omega - \omega') - 2\omega\omega'(1 - \cos \theta) \nonumber \\
       0 &= m\omega -m\omega' - \omega\omega' + \omega\omega'\cos \theta \nonumber \\
       0 &=m\omega - \omega'(m + \omega - \omega\cos\theta) \nonumber \\
       \omega' &= \frac{m\omega}{m + \omega - \omega\cos\theta} \nonumber \\
       \omega' &= \frac{\omega}{1 + \frac{m}{\omega}(1-\cos\theta)} \label{eq:omegaprime}
\end{align}

\subsection*{c}
The $n$-body relativistically invariant phase space is given by \autoref{eq:phasespace}. The two-body integrated version version of this is

\begin{equation}
\int d\Pi_2 = \int \frac{d^3p' d^3k'}{(2\pi)^6} \frac{1}{2E'2\omega'} (2\pi)^4 \delta^{(4)}(k' + p' - k - p).
\end{equation}

First we want to make a variable change, $dk' = (w')^2\sin\theta d\omega' d\theta d\varphi$, where we insert the infinitesimal solid angle $d\Omega = \sin\theta d\theta d\varphi$, which gives $d^3k = (\omega')^2 d\omega' d\Omega$. Inserting this, and using conservation of momentum ($k' + p' = k + p \rightarrow -p' = k' - k -p$)  in the spatial part of the Dirac delta and crossing out equal factors in the numerator and denominator yields

\begin{align}
\int d\Pi_2 &= \int \frac{d^3p'}{(2\pi)^3} \frac{(\omega')^2d\omega' d\Omega}{4\omega' E'} \delta^{(3)} (2\vb{p}') \delta (\omega' + E' - \omega -m) \\
		&= \int \frac{(\omega')^2 d\omega' d\Omega}{(2\pi)^3} \frac{1}{4\omega' E'} \delta (\omega' + E' - \omega - m) .
\end{align}

Now we want to do something with the solid angle. Consider $ d\Omega = \sin\theta d\theta d\varphi$ and notice that there are no $\varphi$ in our expression. We can therefore evaluate the integral, $\int d\phi = 2\pi$. Moreover, $\int_0^\pi \sin\theta d\theta = \int_{-1}^1 -d\cos\theta = \int_1^{-1} d\cos\theta$ allowing the change $\sin\theta d\theta = d\cos\theta$. This should now leave us with

\begin{equation}
\int d\Pi_2 = \int \frac{d\omega' d\cos\theta}{2\pi} \frac{1}{4E'} \delta (\omega' + E' - \omega -m).
\end{equation}

Now to get rid of the remaining Dirac delta. The energy-momentum relation gives
\begin{equation}
p'^2 = E'^2 - \abs{\vb{p}}^2 = m^2 \rightarrow E' = \sqrt{\abs{\vb{p}}^2 + m^2},
\end{equation}

where 

\begin{align}
\vb{p}'^2 &= \vb{p} + \vb{k} - \vb{k}' = \vb{0} + \omega\hat{z} - (\omega\sin\theta, 0, \omega\cos\theta) \\
\abs{\vb{p}'}^2	&= \omega'\sin^2\theta + \omega^2 - 2\omega\omega'\cos\theta + \omega'^2\cos^2\theta \\
						&= \omega^2 + (\omega')^2 - 2\omega\omega'\cos\theta,
\end{align}

so that

\begin{equation}
E' = \sqrt{m^2 + \omega^2 + (\omega')^2 - 2\omega\omega'\cos\theta}.
\end{equation}

This leads to yet another new expression for the phase space

\begin{equation}
\int d\Pi_2 = \int \frac{d\omega' d\cos\theta}{2\pi} \frac{1}{4E'} \delta (\omega' + \sqrt{m^2 + \omega^2 + (\omega')^2 - 2\omega\omega'\cos\theta} - \omega -m).
\end{equation}

In order to proceed we need to use this neat rule
\begin{equation}
\int dx f(x) \delta(g(x)) = \int dx \frac{f(x)}{\abs{g'(x)}}.
\end{equation}

Differentiating the expression inside the Dirac delta function gives us
\begin{equation}
\frac{\partial}{d\omega'} \left(\omega' + \sqrt{m^2 + \omega^2 + (\omega')^2 - 2\omega\omega'\cos\theta} - \omega -m \right)
= 1 + \frac{\omega' - \omega\cos\theta}{E'}.
\end{equation}

Inserting this now gives us
\begin{align}
\int d\Pi_2 &= \int \frac{d\cos\theta}{2\pi} \frac{\omega'}{4E'} \frac{1}{\abs{1 + \frac{\omega' - \omega\cos\theta}{E'}}} \\
			&= \frac{1}{8\pi} \int d\cos \theta \frac{\omega'}{E' + \omega' - \omega\cos\theta} \\
			&= \frac{1}{8\pi} \int d\cos \theta \frac{\omega'}{m + \omega - \omega' + \omega' - \omega\cos\theta} \\
			&= \frac{1}{8\pi} \int d\cos \theta \frac{\omega'}{m + \omega(1 - \cos\theta)}.
\end{align}
Finally, notice that we are not far from \autoref{eq:omegaprime} in the fraction
\begin{align}
\int d\Pi_2 &= \frac{1}{8\pi} \int d\cos \theta \frac{\omega'}{m\left(1 + \frac{\omega}{m}(1 - \cos\theta\right)} \\
 			&= \frac{1}{8\pi} \int d\cos \theta \frac{(\omega')^2}{\omega m}.
\end{align}

Removing the integral signs on both sides and we land on an expression for the two-particle final state phase space,
\begin{equation}
d\Pi_2 = \frac{1}{8\pi} d\cos \theta \frac{(\omega')^2}{\omega m}.
\end{equation}

\subsection*{d}
The infinitesimal cross section is given by \autoref{eq:infinitesimal_cross_section}. Inserting $\abs{\bar{\mathcal{M}}}^2$, $d\Pi_2$, $v= 1$, $E_A = \omega$, $E_B = m$ and dividing through by $d\cos\theta$ yields

\begin{align}
\frac{d\sigma}{d\cos\theta} &= \frac{1}{4\omega m} 2e^4 \left[\frac{m\omega'}{m\omega} + \frac{m\omega}{m\omega'} + \left(1 + \frac{m^2}{m\omega} + \frac{m^2}{m\omega'} \right)^2 - 1 \right] \frac{1}{8\pi} \frac{(\omega')^2}{\omega m} \\
		&= \frac{\pi \alpha^2}{m^2} \left(\frac{\omega'}{\omega} \right)^2 \left[\frac{\omega'}{\omega} + \frac{\omega}{\omega'} + \left( 1 + \frac{m}{\omega} + \frac{m}{\omega'}\right)^2 - 1 \right] \label{eq:finestructurehere} \\
		&= \frac{\pi \alpha^2}{m^2} \left(\frac{\omega'}{\omega} \right)^2 \left[\frac{\omega'}{\omega} + \frac{\omega}{\omega'} + \left( 1  + m (\frac{1}{\omega} - \frac{1}{\omega'})\right)^2 - 1 \right]  \\
		&= \frac{\pi \alpha^2}{m^2} \left(\frac{\omega'}{\omega} \right)^2 \left[\frac{\omega'}{\omega} + \frac{\omega}{\omega'} + \left( 1 - 1 + \cos\theta)\right)^2 - 1 \right] \label{eq:Comptoninserted} \\
		&= \frac{\pi \alpha^2}{m^2} \left(\frac{\omega'}{\omega} \right)^2 \left[\frac{\omega'}{\omega} + \frac{\omega}{\omega'} + \cos^2\theta - 1 \right] \\
		&= \frac{\pi \alpha^2}{m^2} \left(\frac{\omega'}{\omega} \right)^2 \left[\frac{\omega'}{\omega} + \frac{\omega}{\omega'} - \sin^2\theta \right] \label{eq:crossers},
\end{align}

where the fine-structure constant $\alpha = \frac{e^2}{4\pi}$ has been inserted in \autoref{eq:finestructurehere} and Compton's formula for shift in photon wavelength\footnote{$\frac{1}{\omega'} - \frac{1}{\omega} = \frac{1}{m} (1 - \cos\theta)$.} has been inserted in \autoref{eq:Comptoninserted}. In the relativistic limit, where $\omega \to 0$, we see from \autoref{eq:omegaprime} that $\omega' / \omega \to 1$. \autoref{eq:crossers} becomes
\begin{equation}
\frac{d\sigma}{d\cos\theta} = \frac{\pi\alpha^2}{m^2}(1 + \cos^2\theta).
\end{equation}

The full cross section is found by integration
\begin{align}
\int d\sigma = \sigma &= \int_{-1}^1 \frac{\pi\alpha^2}{m^2} (1 + \cos^2 \theta) d\cos\theta 
= \int_{-1}^1 \frac{\pi\alpha^2}{m^2} (1 + u^2) du = \frac{\pi\alpha^2}{m^2}\left[u + \frac{1}{3}u^3 \right]_{-1}^1 \\
	&= \frac{\pi\alpha^2}{m^2} \left(1 + \frac{1}{3} + 1 + \frac{1}{3} \right) \frac{8\pi\alpha^2}{3m^2},
\end{align}
which is the Thompson cross-section for scattering of classical electromagnetic radiation by a free electron.


\end{document}
