\documentclass[]{amsart}

\usepackage{amsmath}
\usepackage{physics}

\title{Problem Set 1 \\ \small{FYS4170}}

\author{Sebastian G. Winther-Larsen}

\begin{document}

\maketitle

\section{Tensor notation}

\subsection{Index form}
Some regular operations on index form.

\begin{align*}
(\nabla S)_i  &= \partial_i S \\
(\nabla\cdot \vb{A})_i &= \partial_i \vb{A} \\
(\nabla \times \vb{A})_i &= \epsilon_{ijk} \partial_j \vb{A} \\
\trace{M} &= M^i_{\ j} \\
M^T = (M_{ij})^T &= M_{ji}
\end{align*}

\subsection{Proof of 3D identities}

\begin{align*}
\nabla\cdot(\nabla\times\vb{A}) = \partial^i \epsilon_{ijk}\partial^j A^k = \epsilon_{ijk} \partial^i \partial^j A^k = 0
\end{align*}
The reasoning behind why this is zero is that when you change two indices in the Levi-Civita symbol, the sign changes. If any of the indices in the other symbols (in accordance with the Levi-Civita change), the sign does not change. One will end up with terms that cancel in pairs.
\begin{align*}
(\nabla\times(\nabla S))_i = \epsilon_{ijk} \partial^j \partial^k S = 0
\end{align*}

The same reasoning works here.

\subsection{Are the equalities valid?}
The following is true,
\begin{align*}
\partial_\mu x^\nu = \delta_\mu^{\ \nu}. 
\end{align*}
Where the delta is a rank two  tensor ("matrix"), with ones along the diagonal.

The following is not true
\begin{equation}
\partial_\mu x^\mu = 1.
\end{equation}
It is quite easy to see why
\begin{align*}
\partial_\mu x^\mu = (\frac{\partial}{\partial x^0}, \frac{\partial}{\partial x^1}, \frac{\partial}{\partial x^2}, \frac{\partial}{\partial x^3} ) \cdot (x^0, x^1, x^2, x^3) = 1 + 1 +1 +1 = 4
\end{align*}

The following is  true
\begin{align*}
\partial^\mu x^\nu = g^{\mu\nu}.
\end{align*}
Here is why
\begin{align*}
\partial^\mu x^\nu = g^{\mu\rho}\partial_{\rho}x^{\nu} = g^{\mu\rho}\delta_\rho^{\ \nu} = g^{\mu\nu}.
\end{align*}

This is OK
\begin{align*}
T_{\alpha \ \gamma}^{\ \beta} = g^{\beta\gamma}T_{\alpha\gamma\mu} = g^{\gamma\beta}T_{\alpha\gamma\mu}.
\end{align*}
The metric raises $\gamma$ and the metric is its own transpose (and inverse). No worries.

This is not OK.
\begin{align*}
T_{\alpha \ \beta}^{\ \beta} = g_{\alpha\mu}g^{\beta\alpha}T^{\mu}_{\ \alpha\beta}.
\end{align*}
Some indices are reused and rules are violated. We need to change some letters
\begin{align*}
T_{\alpha \ \beta}^{\ \beta} = g_{\alpha\mu}g^{\beta\rho}T^{\mu}_{\ \rho\beta}.
\end{align*}

\subsection{Constructing stuff}
All independent Lorentz scalars from two four-vectors A and B:
\begin{align*}
A^\mu B_\mu \quad A^\mu B^\mu \quad B^\mu B^\mu.
\end{align*}

All independent Lorentz scalars from a rank two tensor T is the trace


\end{document}